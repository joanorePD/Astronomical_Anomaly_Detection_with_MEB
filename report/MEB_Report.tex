\documentclass[12pt,a4paper]{article}

\usepackage{graphicx} % Required for inserting images
\usepackage{blindtext}
\usepackage{multicol}
\usepackage{graphicx}
\usepackage{amsfonts} 
\usepackage{amssymb}
\usepackage{amsmath}
\usepackage[a4paper,margin=2cm]{geometry}
\usepackage{amsmath}
\usepackage{algorithm}
\usepackage[noend]{algpseudocode}
\usepackage[breaklinks=true,bookmarks=false]{hyperref}
\usepackage{awesomebox}
\usepackage{fancyhdr}
\usepackage[australian]{babel}
\usepackage{datetime}

\setcounter{page}{1}

\newcommand{\titlestr}{MEB and Anomaly Detection}

\begin{document}

\begin{titlepage}
   \centering
   \includegraphics[width=0.7\textwidth]{unipd_logo.png}

   \vspace{1cm}
   {\LARGE \bf{\titlestr} \par}

   \vspace{.5cm}

   {\LARGE {\it{Optimization for Data Science}} \par}

   \vspace{2cm}

   \bf{
      Granchelli, Cristian\\
    \texttt{ID: 2071977}\\
    \vspace{1cm}
    Kelahan, Cameron\\
    \texttt{ID: 2071947}\\
    \vspace{1cm}
    Orellana Rios, Joan\\
    \texttt{ID: 2081188}
   }\\


\end{titlepage}

%\thispagestyle{empty}

%%%%%%%%% ABSTRACT
\begin{abstract}
   The ABSTRACT is to be in fully-justified italicized text, at the top
   of the left-hand column, below the author and affiliation
   information. Use the word ``Abstract'' as the title, in 12-point
   Times, boldface type, centered relative to the column, initially
   capitalized. The abstract is to be in 10-point, single-spaced type.
   Leave two blank lines after the Abstract, then begin the main text.
   Abstract should be no longer than 300 words.
\end{abstract}

%%%%%%%%% BODY TEXT
\section{Introduction}

%-------------------------------------------------------------------------
\section{Minimum Enclosing Balls}

[Here is where I'd do the transition between the formulation of the problem in the introduction and how we intend to solve it using MEB]

\subsection{Description of MEB}

We can describe the Minimum Enclosing Ball problem in the following way: given a set of points \(\mathcal{A} = a_1, ..., a_n\) in \(d\) dimensions, the objective is to find the smallest ball \(B^n(c, \rho) \) containing all points of $\mathcal{A}$, such that:

\begin{equation}
\mathcal{A} \subseteq B^n(c, \rho)
\end{equation}\\
where

\begin{equation}
B^n(c, \rho) = \{x \in \mathbb{R}^n \, | \, \|x - c\| \leq \rho\}
\end{equation}\\
where \( c \in \mathbb{R}^n \) and \( \rho \in \mathbb{R} \) are the decision variables. We define \( c \) as the centre of the minimum enclosing ball and \( \rho \) as the radius. A sphere \(B^n(c, \rho)\) \(\subset \mathbb{R}^n\) is called an enclosing ball of \(\mathcal{A} \) if and only if \((1)\) holds.
\\

The problem of calculating an approximation to the minimum enclosing ball  of $\mathcal{A}$ can be denoted as MEB($\mathcal{A}$), which can be computed by solving the following optimization problem:

\begin{align*}
\text{($\mathcal{P}_1$)} \quad & \min_{c, \rho} \rho \\
\text{subject to} \quad & a_i - c \leq \rho, \quad i = 1, \ldots, m\\
\end{align*}

We obtain a second optimization problem $\mathcal{P}_2$ with smooth and convex quadratic constraints, by setting \(\gamma := \rho^2 \) and squaring the constraints of $\mathcal{P}_1$.

\begin{align*}
\text{($\mathcal{P}_2$)} \quad & \min_{c, \gamma} \gamma \\
\text{subject to} \quad & (a_i)^T a_i - 2(a_i)^T c + c^T c \leq \gamma, \quad i = 1, \ldots, m
\end{align*}

We can derive the lagrangian dual from $\mathcal{P}_2$ by adding Lagrange multiplier to each constraint.

\newpage
%-------------------------------------------------------------------------
\subsection{References}

List and number all bibliographical references in 9-point Times,
single-spaced, at the end of your paper. When referenced in the text,
enclose the citation number in square brackets, for
example~\cite{Authors14}.  Where appropriate, include the name(s) of
editors of referenced books.


{\small
\bibliographystyle{ieee_fullname}
\bibliography{egbib}
}

\end{document}
